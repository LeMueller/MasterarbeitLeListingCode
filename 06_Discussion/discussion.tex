\chapter{Zusammenfassung}

Durch die Studie werden der Lerneffekt und die Benutzererfahrung der VR Übung untersucht. In diesem Kapitel werden die Erreichung der Ziele, die getroffenen Probleme während der Entwicklung und die Aussicht der WebVR Technik diskutiert.

\section{Erreichung der Ziele}

Im Kapitel Einleitung werden sechs Ziele für das Projekt gesetzt. Durch die Evaluation des Projektes wird bestimmt, ob diese Ziele erreicht werden. Hier werden die Ergebnisse für alle Ziele aufgelistet:

\begin{enumerate}[labelsep=1ex]
	\renewcommand{\labelenumi}{\textbf{Z\theenumi.}}
	\item Ein Unterricht mit VR Übung wird in Moodle erstellt. Der Lerneffekt ist deutlich. Alle neun Versuchspersonen haben den Test über Infusionsvorbereitung durchschnittlich mit der Note 91 von 100 bestanden. Im Durchschnitt erwarb eine Versuchsperson bei dem Test nach der Übung 22 Punkte mehr als dem Test vor der VR Übung.
	
	\item Die VR Übung kann auf PC, Smartphone, Samsung Gear VR und HTC Vive durchgeführt werden. Mit PC, Smartphone und HTC Vive wird die Studie gemacht. Durchschnittlich werden 4,1 von 5 Punkte geben, wenn die Frage gestellt wird, ob man das Gefühl hat, wirklich eine Infusion vorzubereiten.
	
	\item Keine Installation oder Aktualisierung der WebVR Applikation wird gefordert. Die WebVR Applikation kann direkt durch die eingegebene URL im Browser aufgerufen werden.
	
	\item Ring Zeiger wird auf PC, Smartphone, Gear VR eingesetzt. Raycaster wird für Gear VR Controller eingesetzt. Die Funktion der Hände wird für HTC Vive eingesetzt.
	
	\item Die Zielobjekte können während der Übung mühelos gefunden werden. Das Whiteboard bietet die Hinweise für den nächsten Schritt, was von den Versuchspersonen für sehr hilfreich befunden wird. Das Feedback der Objekte entspricht den Erwartungen der Versuchspersonen.
	
	\item Die VR Übung wird durch die Links in dem Unterricht aufgerufen. Und durch den gezeigten Link auf dem Whiteboard in der VR Umgebung können die Versuchspersonen wieder zurück zum Unterricht geleitet werden. Alle Versuchspersonen finden, dass die VR Übung sehr gut mit dem Unterricht verbunden wird.
	
\end{enumerate}

Zusammenfassend kann behauptet werden, dass das Projekt die Ziele erreicht hat.

\section{Probleme der WebVR Applikation}
Obwohl die Ziele erreicht werden, sollen die Probleme bei der WebVR Applikation nicht ignoriert werden. Hier werden die Probleme aus den Aspekten Entwicklung und Einsetzung erzählt.

\subsection{Entwicklung}

Bei dem Framework A-Frame fehlt es zurzeit noch an wichtigen built-in Funktionen und stabiler Entwicklungsumgebung, was von den Game Engines (Unity und Unreal Engine) angeboten wird, deswegen werden Probleme getroffen während der Entwicklung. 

\begin{itemize}
    \item \textbf{Mangel an built-in Funktionen}: Bei A-Frame fehlen wichtig Funktionen wie Kollisionserkennung und freier Fall, die von Unity und Unreal Engine als buit-in Funktionen geboten werden. Das führt zu hohem Zeitaufwand und schlechter Qualität der Applikation, wenn nicht genügend Zeit zu Verfügung steht.
    
    \item \textbf{Schlechte Erfahrung bei Debug}: Breakpoint ist ein hilfreiches Debug-Werkzeug für Software Entwicklung. In Browser wird auch das Breakpoint geboten. Allerdings passt es nicht an der Entwicklung von HTC Vive. Wenn die WebVR Applikation sich durch breakpoint pausiert, wird die VR Umgebung nach dem Übersprung des Breakpints nicht mehr in HMD gezeigt, so dass das Debuggen nur mit der Funktion \glqq console.log() \grqq\ durchgeführt werden kann, was zu hohem Zeitaufwand führt.
    
\end{itemize}

\subsection{Einsetzung}

Die Installation und Aktualisierung für webbasierte Applikation sind kein Thema. Jedoch gibt es noch unüberbrückbare Probleme bei der Einsetzung.

\begin{itemize}
    \item \textbf{Abhängigkeit von Browsers}: Browser spielen eine wichtige Rolle bei webbasierten Applikationen. Das Endergebnis der Applikation auf unterschiedlichen Browsern könnte anders sein, deswegen wird die WebVR Applikation für unterschiedlichen Browser angepasst. Allerdings könnte die Aktualisierung des Browsers zur Änderung des Endergebnisses führen. Deshalb soll die VR Übung regelmäßig gepflegt werden.
    
    \item \textbf{Begrenzte Größe der Applikation}: Die ganze WebVR Applikation wird bei jedem Aufruf heruntergeladen, deswegen entscheidet die Größe der Applikation über die Geschwindigkeit, die Applikation zu laden. Aus diesem Grund kann die WebVR Applikation nicht groß sein, was die Qualität der Modelle und die Menge der Soundeffekte beschränkt.
\end{itemize}

\section{Ausblick}

Obwohl wegen der technischen Probleme die VR Übung nicht optimal ist, werden alle Ziele dieses Projektes erreichtet. Außerdem hat die VR Übung laut der Studie einen sehr guten Lerneffekt.

Zurzeit sind einige Probleme nicht lösbar. Allerdings werden solche Probleme in der Zukunft wegen der Entwicklung der Infrastruktur (höhere Internet-Geschwindigkeit) und die WebVR Technik (bessere Werkzeuge) behoben.

Webbasierte Applikation ist den Trend der Entwicklung, und wird schon im Bereich E-Learning weit verbreitet. Als neuer Teil der webbasierten Technik werden WebVR und WebAR in mehreren Bereichen eingesetzt.

