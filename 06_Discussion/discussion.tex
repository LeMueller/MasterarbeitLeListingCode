\chapter{Diskussion}

Durch die Studie werden der Lerneffekt und die Benutzererfahrung der VR Übung untersucht. In diesem Kapitel wird zunächst die Evaluationsdaten von Kapitel 5 ausgewertet. Anschließend werden die Erreichung der Ziele, die getroffenen Probleme während der Entwicklung und die Aussicht der WebVR Technik diskutiert.

\section{Auswertung der Evaluationsdaten}

Durch die Analyse der gesammelten Daten, werden die Erfolge und die Probleme des Projektes gefunden.

\subsection{Erfolge}

\begin{itemize}
    \item \textbf{Lerneffekt der VR Übung}
    
    Der Fortschritt nach der VR Übung ist deutlich, 10 bis 29 Punkte werden mehr erworben. Es passiert während der Studie zwei Mal, dass die Versuchsperson bei dem erstmaligen Test schon alle Fragen richtig beantwortet hat. Allerdings habe sie die Erfahrung, dass sie nicht 100\% sicher für die Antworten bei dem erstmaligen Test ist. Nach der VR Übung kam das Selbstbewusstsein, dass sie alle Punkte kriegen konnten.
    
    \item \textbf{Hinweise auf dem Whiteboard}
    
    Alle Versuchspersonen haben die Hinweise benutzt und glauben, dass die Hinweise sehr hilfreich sind.
    
    \item \textbf{Integration der VR Übung in den Unterricht}
    
    Alle Versuchspersonen sind der Meinung, dass die VR Übung sehr gut in den Unterricht integriert wird. Inhaltlich wiederholt die VR Übung die Kenntnisse. Äußerlich gilt die VR Übung als eine Aktivität des Unterrichts.
    
    \item \textbf{Geräusche Feedbacks}
    
    Es kann passieren, dass es keine Reaktion des Zielobjektes nach der Interaktion gibt, deswegen gelten die Geräusche als zuverlässiges Feedback für jede Interaktion.
    
\end{itemize}

\subsection{Probleme}
\begin{itemize}
    \item \textbf{Objekte schwer zu finden (Smartphone > PC > Vive)}
    
    Das Problem erscheint auf allen drei Geräten. Das Problem auf Smartphone und PC ist dabei gravirender als auf HTC Vive. Es gibt zwei Gründe für die Schwierigkeit, die Objekte zu finden.
    
    Der erste Grund ist, dass die Umgebung und die Objekte für die Versuchspersonen, die keine Pflege-Studenten sind, fremd sind. Es passiert mehrere Male, dass die Versuchspersonen nicht wissen, wo die Krankenakte und die Einmalhandschuhe abgelegt werden, weil die Krankenakte geschlossen ist und der Spender für Einmalhandschuhe nicht bekannt ist.
    
    Der zweite Grund ist, dass die Ansicht der Smartphones klein ist. Die Hängeschränke sind nicht in der Sicht des Smartphones nach der Initialisierung.
    
    Um das Problem zu lösen, sollen die Umgebung und die Objekte in den Unterricht als Vorwissen vorgestellt werden, oder soll eine Unterweisung über der Umgebung und den Objekten in der VR Umgebung vor dem Start der Übung geboten werden.
  
    \item \textbf{Fehlen des aktiven Hinweises}
    
    In der VR Umgebung werden die Hinweise durch das Whiteboard dargestellt. Allerdings ist es ein passives Angebot. Das heißt, dass die Hinweise ignoriert werden könnten, wenn der Benutzer nicht zum Whiteboard schaut. Deswegen passiert es, dass sich die Versuchsperson lange Zeit auf ein falsches Objektes konzentriert, weil außer dem Whiteboard keine andere aktive Hinweise eingefügt wurde.
    
    Um das Problem zu lösen, soll ein aktiver Hinweis eingesetzt werden, d.h. direkt vor den Augen in den richtigen Zeiträumen darzustellen. Der Hinweis kann z.B. nur ein Vorschlag sein, das Whiteboard anzuschauen.
    
    \item \textbf{Die Entfernung der Einmalhandschuhe nicht deutlich}
    
    Nach der Desinfektion der Arbeitsfläche werden die Merkmale der Einmalhandschuhe (Einmalhandschuhe Modell vor den Augen in PC und Smartphone und blauer Farbe auf die Hände in Vive) entfernt. Aber die Entfernung ist nicht deutlich, weil es keine Animation dafür hinzugefügt worden ist, sodass manche Versuchspersonen nicht merken konnten, wann die Handschuhe entfernt worden sind.
    
    Mit einer Animation für die Entfernung der Handschuhe kann das Problem behoben werden.
    
    \item \textbf{Fehlen des Feedback nach der Prüfung für Infusionsflasche und Infusionsbesteck mit PC und Smartphone}
    
    Wenn PC und Smartphone benutzt werden, gelten die Bewegungen der Infusionsflasche und der Infusionsbesteck als die Feedbacks der Prüfung anstelle des Haken auf den geprüften Positionen. Allerdings war ein solches Feedback nicht deutlich, sodass die Schwerpunkte der Überprüfung nicht gemerkt wurden.
    
    Um das Problem zu lösen, sollen die Haken nach der entsprechenden Prüfung dargestellt werden.
\end{itemize}



\section{Erreichung der Ziele}

Im Kapitel Einleitung werden sechs Ziele für das Projekt gesetzt. Durch die Evaluation des Projektes wird bestimmt, ob diese Ziele erreicht werden. Hier werden die Ergebnisse für alle Ziele aufgelistet:

\begin{enumerate}[labelsep=1ex]
	\renewcommand{\labelenumi}{\textbf{Z\theenumi.}}
	\item Ein Unterricht mit VR Übung wird in Moodle erstellt. Der Lerneffekt ist hoch. Alle neun Versuchspersonen haben den Test über Infusionsvorbereitung durchschnittlich mit der Note 91 von 100 bestanden. Im Durchschnitt erwarb eine Versuchsperson bei dem Test nach der Übung 22 Punkte mehr als dem Test vor der VR Übung.
	
	\item Die VR Übung kann auf PC, Smartphone, Samsung Gear VR und HTC Vive durchgeführt werden. Mit PC, Smartphone und HTC Vive wird die Studie gemacht. Durchschnittlich werden 4,1 von 5 Punkte für die Frage angegeben, ob man das Gefühl hat, wirklich eine Infusion vorzubereiten.
	
	\item Keine Installation oder Aktualisierung der WebVR Applikation wird gefordert. Die WebVR Applikation kann direkt durch die eingegebene URL im Browser aufgerufen werden.
	
	\item Ring Zeiger wird auf PC, Smartphone, Gear VR eingesetzt. Raycaster wird für Gear VR Controller eingesetzt. Die Funktion der Hände wird für HTC Vive eingesetzt.
	
	\item Die Zielobjekte können während der Übung mühelos gefunden werden. Das Whiteboard bietet die Hinweise für den nächsten Schritt, was von den Versuchspersonen für sehr hilfreich befunden wird. Das Feedback der Objekte entspricht den Erwartungen der Versuchspersonen.
	
	\item Die VR Übung wird durch die Links in dem Unterricht aufgerufen. Durch den gezeigten Link auf dem Whiteboard in der VR Umgebung können die Versuchspersonen wieder zurück zum Unterricht geleitet werden. Alle Versuchspersonen finden, dass die VR Übung sehr gut mit dem Unterricht verbunden ist.
	
\end{enumerate}

Zusammenfassend kann festgestellt werden, dass das Projekt die Ziele erreicht hat.

\section{Probleme der WebVR Applikation}
Obwohl die Ziele erreicht wurden, sollen die Probleme bei der WebVR Applikation nicht ignoriert werden. Hier werden die Probleme aus den Aspekten Entwicklung und Einsetzung erörtert.

\subsection{Entwicklung}

Bei dem Framework A-Frame fehlt es zurzeit noch an wichtigen built-in Funktionen und einer stabilen Entwicklungsumgebung, was von den Game Engines (Unity und Unreal Engine) angeboten wird, deswegen werden Probleme getroffen während der Entwicklung. 

\begin{itemize}
    \item \textbf{Mangel an built-in Funktionen}: Bei A-Frame fehlen wichtig Funktionen wie Kollisionserkennung und freier Fall, die von Unity und Unreal Engine als buit-in Funktionen geboten werden. Das führt zu einem hohen Zeitaufwand. Wenn keine genügende Zeit zur Verfügung steht, kann die Applikation nur mit mittelmäßiger Qualität entwickelt werden.
    
    \item \textbf{Schlechte Erfahrung bei Debug}: Breakpoint ist ein hilfreiches Debug-Werkzeug für Software Entwicklung. In Browser wird auch das Breakpoint geboten. Allerdings passt es nicht an der Entwicklung von HTC Vive. Wenn die WebVR Applikation sich durch breakpoint pausiert, wird die VR Umgebung nach dem Übersprung des Breakpints nicht mehr in HMD gezeigt, so dass das Debuggen nur mit der Funktion {\fontfamily{qcr}\selectfont console.log()} durchgeführt werden kann, was zu hohem Zeitaufwand führt.
    
\end{itemize}

\subsection{Einsetzung}

Die Installation und Aktualisierung für webbasierte Applikation sind leicht durchzuführen. Jedoch gibt es noch unüberbrückbare Probleme bei der Einsetzung.

\begin{itemize}
    \item \textbf{Abhängigkeit von Browsers}: Browser spielen eine wichtige Rolle bei webbasierten Applikationen. Das Endergebnis der Applikation auf unterschiedlichen Browsern könnte anders sein, deswegen wird die WebVR Applikation für unterschiedlichen Browser angepasst. Allerdings könnte die Aktualisierung des Browsers zur Änderung des Endergebnisses führen. Deshalb soll die VR Übung regelmäßig gepflegt werden.
    
    \item \textbf{Begrenzte Größe der Applikation}: Die ganze WebVR Applikation wird bei jedem Aufruf heruntergeladen, deswegen entscheidet die Größe der Applikation über die Internet-Geschwindigkeit, die Applikation zu laden. Aus diesem Grund kann die WebVR Applikation nicht zu groß sein, was die Qualität der Modelle und die Menge der Soundeffekte beschränkt.
\end{itemize}

\section{Ausblick}

Obwohl wegen der technischen Probleme die VR Übung nicht optimal ist, werden alle Ziele dieses Projektes erreichtet. Darüber hinaus hat die VR Übung laut der Studie einen sehr guten Lerneffekt.

Zurzeit sind einige Probleme nicht lösbar. Allerdings werden solche Probleme in der Zukunft wegen der Entwicklung der Infrastruktur (höhere Internet-Geschwindigkeit) und die WebVR Technik (bessere Werkzeuge) behoben.

Webbasierte Applikation ist den Trend der Entwicklung, und wird schon im Bereich E-Learning weit verbreitet. Als neuer Teil der webbasierten Technik werden WebVR und WebAR in mehreren Bereichen eingesetzt.

