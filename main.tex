\documentclass[ncs,12pt,oneside,german,listof=totoc]{YY_Styles/abschlussarbeit}

\usepackage{gensymb}
\usepackage{graphicx}

\usepackage{tgbonum}

\usepackage{a4}
\usepackage{etex}
\usepackage[utf8]{inputenc}
\usepackage[T1]{fontenc}

\usepackage{authoraftertitle}
\usepackage{titling}

\usepackage{times}
%\usepackage[german,english]{babel}
\usepackage{relsize}
\usepackage{lmodern}

\usepackage{fancyhdr}
\usepackage{fancybox}

%
% Graphics
%
\usepackage{wrapfig}
\setlength{\intextsep}{0cm plus1cm minus1cm}
 
\usepackage[tight]{subfigure}
\usepackage{graphicx}
\usepackage{pdfpages}
\usepackage{epstopdf}
\usepackage{wallpaper}


%
% Math
%
\usepackage{mathptmx}
\usepackage{amsmath}
\usepackage{amsfonts}
\usepackage{amssymb}
\usepackage{stmaryrd}
%
% Bibliography and References
%
\usepackage[longnamesfirst]{natbib}
\usepackage{citeref}
\usepackage{varioref}

% ,pagebackref fuehrte hier zu einem Haenger beim Erzeugen\ldots
\usepackage[printonlyused,withpage]{acronym}

%
% Multimedia
%
\usepackage[3D]{movie15}

%
% Source-Code
%
\usepackage[ruled]{algorithm}
\usepackage{algorithmicx}
\usepackage{algpseudocode}
\usepackage{listings}


%
% Tools
%
\usepackage{import}
\usepackage{url}
\usepackage[colorinlistoftodos,shadow]{todonotes}
\usepackage{enumitem}

%\pagestyle{fancy}

\definecolor{light-gray}{gray}{0.80}
\definecolor{mygray}{gray}{0.50}
\definecolor{myblack}{gray}{0.0}

%\fancypagestyle{plain}{%
%\fancyhf{}
%\fancyhead[LE]{\color{myblack}\thepage}
%\fancyhead[RO]{\color{myblack}\thepage}
%\fancyhead[RE]{\slshape{\leftmark}}
%\fancyhead[LO]{\slshape{\rightmark}}
%%\fancyfoot[LO,RE]{\color{mygray}\today}
%\fancyfoot[C]{}
%\fancyfoot[LE,RO]{}
%}
%\fancyhead[LE]{\color{myblack}\thepage}
%\fancyhead[RO]{\color{myblack}\thepage}
%\fancyhead[RE]{\slshape{\leftmark}}
%\fancyhead[LO]{\slshape{\rightmark}}
%%\fancyfoot[LO,RE]{\color{mygray}\today}
%\fancyfoot[C]{}
%\fancyfoot[LE,RO]{}

\addtolength{\headheight}{3pt}
\renewcommand{\chaptermark}[1]{\markboth{\color{myblack}Chapter \thechapter{}\
#1}{}}
\renewcommand{\sectionmark}[1]{\markright{\color{myblack}\thesection{}\ #1}}
\renewcommand{\headrule}{{\color{myblack}%
\hrule width\headwidth height\headrulewidth \vskip-\headrulewidth}}

\usepackage{hyperref}
\hypersetup{
	unicode,%
	breaklinks,%
	colorlinks=true,%
	linkcolor=black,%
	linktoc=all,%
	citecolor=black,%
	filecolor=black,%
	urlcolor=black%
	}
\usepackage[all]{hypcap}

%
% Layout haerter einstellen
%
\clubpenalty = 10000
\widowpenalty = 10000
\displaywidowpenalty = 10000
\tolerance=9000

%
% Coole Items für Checklisten
%
\newcommand{\gooditem}{\item[\ding{51}]\hspace{4pt}}
\newcommand{\baditem}{\item[\ding{53}]\hspace{4pt}}

\usepackage{booktabs}

\graphicspath{ {./images/} }

% roman number
\newcommand{\RomanNumeralCaps}[1]
    {\MakeUppercase{\romannumeral #1}}

% code
\usepackage{color}
\definecolor{lightgray}{rgb}{0.95, 0.95, 0.95}
\definecolor{darkgray}{rgb}{0.4, 0.4, 0.4}
%\definecolor{purple}{rgb}{0.65, 0.12, 0.82}
\definecolor{editorGray}{rgb}{0.95, 0.95, 0.95}
\definecolor{editorOcher}{rgb}{1, 0.5, 0} % #FF7F00 -> rgb(239, 169, 0)
\definecolor{editorGreen}{rgb}{0, 0.5, 0} % #007C00 -> rgb(0, 124, 0)
\definecolor{orange}{rgb}{1,0.45,0.13}		
\definecolor{olive}{rgb}{0.17,0.59,0.20}
\definecolor{brown}{rgb}{0.69,0.31,0.31}
\definecolor{purple}{rgb}{0.38,0.18,0.81}
\definecolor{lightblue}{rgb}{0.1,0.57,0.7}
\definecolor{lightred}{rgb}{1,0.4,0.5}
\usepackage{upquote}
\usepackage{listings}
% CSS
\lstdefinelanguage{CSS}{
  keywords={color,background-image:,margin,padding,font,weight,display,position,top,left,right,bottom,list,style,border,size,white,space,min,width, transition:, transform:, transition-property, transition-duration, transition-timing-function},	
  sensitive=true,
  morecomment=[l]{//},
  morecomment=[s]{/*}{*/},
  morestring=[b]',
  morestring=[b]",
  alsoletter={:},
  alsodigit={-}
}

% JavaScript
\lstdefinelanguage{JavaScript}{
  morekeywords={typeof, new, true, false, catch, function, return, null, catch, switch, var, let, const, if, in, while, do, else, case, break},
  morecomment=[s]{/*}{*/},
  morecomment=[l]//,
  morestring=[b]",
  morestring=[b]'
}

\lstdefinelanguage{HTML5}{
  language=html,
  sensitive=true,	
  alsoletter={<>=-},	
  morecomment=[s]{<!-}{-->},
  tag=[s],
  otherkeywords={
  % General
  >,
  % Standard tags
	<!DOCTYPE,
  </html, <html, <head, <title, </title, <style, </style, <link, </head, <meta, />,
	% body
	</body, <body,
	% Divs
	</div, <div, </div>, 
	% Paragraphs
	</p, <p, </p>,
	% scripts
	</script, <script,
  % More tags...
  <canvas, /canvas>, <svg, <rect, <animateTransform, </rect>, </svg>, <video, <source, <iframe, </iframe>, </video>, <image, </image>, <header, </header, <article, </article
  },
  ndkeywords={
  % General
  =,
  % HTML attributes
  charset=, src=, id=, width=, height=, style=, type=, rel=, href=,
  % SVG attributes
  fill=, attributeName=, begin=, dur=, from=, to=, poster=, controls=, x=, y=, repeatCount=, xlink:href=,
  % properties
  margin:, padding:, background-image:, border:, top:, left:, position:, width:, height:, margin-top:, margin-bottom:, font-size:, line-height:,
	% CSS3 properties
  transform:, -moz-transform:, -webkit-transform:,
  animation:, -webkit-animation:,
  transition:,  transition-duration:, transition-property:, transition-timing-function:,
  }
}

\lstdefinestyle{htmlcssjs} {%
  % General design
%  backgroundcolor=\color{editorGray},
  basicstyle={\footnotesize\ttfamily},   
  frame=b,
  % line-numbers
  xleftmargin={0.75cm},
  numbers=left,
  stepnumber=1,
  firstnumber=1,
  numberfirstline=true,	
  % Code design
  identifierstyle=\color{black},
  keywordstyle=\color{blue}\bfseries,
  ndkeywordstyle=\color{editorGreen}\bfseries,
  stringstyle=\color{editorOcher}\ttfamily,
  commentstyle=\color{brown}\ttfamily,
  % Code
  language=HTML5,
  alsolanguage=JavaScript,
  alsodigit={.:;},	
  tabsize=2,
  showtabs=false,
  showspaces=false,
  showstringspaces=false,
  extendedchars=true,
  breaklines=true,
  % German umlauts
  literate=%
  {Ö}{{\"O}}1
  {Ä}{{\"A}}1
  {Ü}{{\"U}}1
  {ß}{{\ss}}1
  {ü}{{\"u}}1
  {ä}{{\"a}}1
  {ö}{{\"o}}1
}
%


% Umgang mit Schusterjungen und Hurenkindern
% komplett verbieten mit:
% \clubpenalty = 10000
% \widowpenalty = 10000
% \displaywidowpenalty = 10000

% oder manuell Absätze kürzer
% \looseness=-1
% oder laenger
% \looseness=1
% umbrechen lassen\ldots

%
% START
%
\begin{document}

%
% Frontmatter
% ===========
%
% Kein Frontmatter, da dann der Titel noch einmal generiert wird...
%\begin{frontmatter}
\renewcommand{\baselinestretch}{1.08}
% Titelseite
\begin{titlepage}

\title{Webbasiertes virtuelles Training mit Anbindung an Lern-Management-System}
\author{Le Zhang}
\date{\today}

\dimendef\prevdepth=0
\begin{center}
\thispagestyle{empty}

{\fontsize{24.88}{28}\usefont{OT1}{cmss}{pifont}{n}
 \thetitle\\[0.5cm]
}
\fontsize{12}{12}
\usefont{OT1}{cmss}{pifont}{n}
\vspace{1cm}
Masterarbeit\\
im Studiengang Interdisziplinäre Medienwissenschaft\\
an der Linguistik und Literaturwissenschaft Fakultät\\
der Universität Bielefeld\\

\vspace{1.5cm} \today

\vspace{1cm}
vorgelegt von\\
\theauthor

\vspace{3cm}
Betreuer:\\
Dr. Thies Pfeiffer\\
CITEC, Technische Fakultät, Universität Bielefeld\\
Inspiration 1\\
33619 Bielefeld\\
\vspace{.2cm}
Paul John\\
Zentrum für Lehren und Lernen (ZLL) / eLearning / Medien, Universität Bielefeld\\
Universitätsstraße 25\\
33615 Bielefeld


\vspace{2cm}
{\small \theauthor, Anschrift}
\end{center}
\end{titlepage}

\thispagestyle{empty}
\newpage
\thispagestyle{empty}
\cleardoublepage
\thispagestyle{empty}
\input{second_page.tex}
%
\renewcommand{\baselinestretch}{1.08}
%
%\newpage               % leere Seite nach der Titelseite
%\thispagestyle{empty}  % und bitte auch ohne Seitenzahlen

% Im Vorspann bitte römische Zahlen
\pagenumbering{Roman} 
%
\subimport{.}{summary.tex}
%
% Welche Level sollen im Inhaltsverzeichnis dargestellt werden?
% Hier Tiefe 3 bei sections
\setcounter{secnumdepth}{3}
\setcounter{tocdepth}{2}
\tableofcontents
%
%\newpage
%\thispagestyle{empty}
%
%\newpage
%\thispagestyle{empty}
%\cleardoublepage
%
\cleardoublepage
\phantomsection \label{listoffigures}
%\addcontentsline{toc}{chapter}{Abbildungsverzeichnis}
\listoffigures
%Das Abbildungsverzeichnis ist nur dann sinnvoll, wenn es viele Abbildungen gibt, die wichtige Inhalte vermitteln. Insbesondere sollte das bei Bachelorarbeiten nicht übertrieben werden, sonst gibt es letztlich mehr Seiten Struktur als Inhalt.
%
\cleardoublepage
\phantomsection \label{listoftables}
%\addcontentsline{toc}{chapter}{Tabellenverzeichnis}
\listoftables
%
\cleardoublepage
% Switch to standard font for acronyms
%\renewcommand{\aclabelfont}[1]{\normalfont{\normalsize{#1}}\hfill}
%\phantomsection \label{listofacronyms}
%\addcontentsline{toc}{chapter}{Abkürzungsverzeichnis}
%\markboth{Abkürzungsverzeichnis}{Abkürzungsverzeichnis}
%\subimport{.}{acronyms.tex}
%
%\cleardoublepage
%\phantomsection \label{acknowledgement}
%\addcontentsline{toc}{chapter}{Danksagung}
%\subimport{.}{acknowledgement.tex}

%\end{frontmatter}

\begin{mainmatter}

\subimport{01_Introduction/}{introduction.tex}
\subimport{02_RelatedWork/}{relatedwork.tex}
\subimport{03_Concept/}{concept.tex}
\subimport{04_Implementation/}{implementation.tex}
\subimport{05_Evaluation/}{evaluation.tex}
\subimport{06_Discussion/}{discussion.tex}

\cleardoublepage
\phantomsection \label{bibliography}
\addcontentsline{toc}{chapter}{Literaturverzeichnis}
\bibliographystyle{newapa}
\bibliography{literature}

\end{mainmatter}

\begin{backmatter}
\part*{Appendix\markboth{Appendix}{}}
\begin{appendix}
\refstepcounter{chapter}
\nocite{*}
\addcontentsline{toc}{chapter}{Anhang}
\subimport{XX_Appendix/}{appendix.tex}
\end{appendix} 
\end{backmatter}
\end{document}
